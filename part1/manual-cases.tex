\documentclass{article}

\title{Part 1 Manual Specification of Test Cases}

\author{Ambrose Bonnaire-Sergeant (20350292)}

\begin{document}

\maketitle

% Prerequisites/assumptions
% Test Data (variables, fields etc.)

\tableofcontents

\section{Document Description}

This document describes some manually generated test cases for testing
Moodle.

I have concentrated on these pages:

\begin{itemize}
  \item Add new user page
  \item Edit user profile page
  \item Edit password page
\end{itemize}

Tests that validate data entry conditions are the main theme in this document.

Test identifiers are organised as follows:

\begin{description}
  \item[DENU] Data entry (for) new user page
  \item[DEEU] Data entry (for) edit user page
\end{description}

Identifiers end with \emph{S} for success cases or \emph{F} for
failure/negative cases.

I chose to concentrate on negative test cases because they are easiest
to test. Usually an input validation error results in an error message,
which would be simple to search for and verify using an automated
testing tool like Selenium.

%%%%%%%%%%%%%%%%%%%%%%%%%%%%%%%%%%%%%%%%%%%%%%%%%%%%%%%%%%%%%%%%%
%% TEST NEW USER PAGE
%%%%%%%%%%%%%%%%%%%%%%%%%%%%%%%%%%%%%%%%%%%%%%%%%%%%%%%%%%%%%%%%%

\section*{DENUS1 - Adding a user succeeds}

\subsection*{Prerequisites}

\begin{itemize}
  \item Logged in as the main administrator
  \item Currently at the Moodle home page
\end{itemize}

\subsection*{Test Data}

\begin{itemize}
  \item Valid username
  \item Valid newpassword
  \item Valid firstname
  \item Valid email
  \item Valid city
  \item Valid country
\end{itemize}

\subsection*{Steps to carry out the test}

\begin{itemize}
  \item Click the button ``Add a new user'' under the User and then
    the Account categories.
  \item Fill form with the minimum amount of data: username, newpassword,
    firstname, email, city and country.
  \item Click Create User button.
  \item Verify that no errors appear on the page.
\end{itemize}

\section*{DENUF2 - Username must be valid}

Same as DENUS1, except an invalid username is provided.


Verify that the correct error has been printed.

\section*{DENUF3a - Password must be valid}

Same as DENUS1, except an invalid password is provided.

The password 
\begin{itemize}
 \item does not contain at least 1 digit
 \item does not contain a non-alphanumeric character
 \item does not contain an upper case letter
 \item is not at least 8 characters long
\end{itemize}

Verify that the correct error has been printed, and that
irrelevant password-related errors are not printed.

\section*{DENUF3b - Password must be valid}

Same as DENUS1, except an invalid password is provided.

The password
\begin{itemize}
  \item does not contain at least 1 digit
  \item does not contain a non-alphanumeric character
  \item does not contain an upper case letter
  \item IS at least 8 characters long
\end{itemize}

Verify that the correct error has been printed, and that
irrelevant password-related errors are not printed.

\section*{DENUF3c - Password must be valid}

Same as DENUS1, except an invalid password is provided.

The password
\begin{itemize}
  \item does not contain at least 1 digit
  \item does not contain a non-alphanumeric character
  \item DOES contain an upper case letter
  \item IS at least 8 characters long
\end{itemize}

Verify that the correct error has been printed, and that
irrelevant password-related errors are not printed.

\section*{DENUF3d - Password must be valid}

Same as DENUS1, except an invalid password is provided.

The password
\begin{itemize}
  \item does not contain at least 1 digit
  \item DOES contain a non-alphanumeric character
  \item DOES contain an upper case letter
  \item IS at least 8 characters long
\end{itemize}

Verify that the correct error has been printed, and that
irrelevant password-related errors are not printed.

\section*{DENUF3e - Password must be valid}

Same as DENUS1, except an invalid password is provided.

The password
\begin{itemize}
  \item DOES contain at least 1 digit
  \item does not contain a non-alphanumeric character
  \item DOES contain an upper case letter
  \item IS at least 8 characters long
\end{itemize}

Verify that the correct error has been printed, and that
irrelevant password-related errors are not printed.

\section*{DENUF3f - Password must be valid}

Same as DENUS1, except an invalid password is provided.

The password
\begin{itemize}
  \item DOES contain at least 1 digit
  \item DOES contain a non-alphanumeric character
  \item DOES contain an upper case letter
  \item is not at least 8 characters long
\end{itemize}

Verify that the correct error has been printed, and that
irrelevant password-related errors are not printed.

\section*{DENUF4 - Email must be valid}

Same as DENUS1, except an invalid email is provided.
Verify that the correct error has been printed.

\section*{DENUF5 - Username must not be missing}

Same as DENUS1, except no username is provided.
Verify that the correct error has been printed.

\section*{DENUF6 - Password must not be missing}

Same as DENUS1, except no password is provided.
Verify that the correct error has been printed.

\section*{DENUF7 - Firstname must not be missing}

Same as DENUS1, except no firstname is provided.
Verify that the correct error has been printed.

\section*{DENUF8 - Surname must not be missing}

Same as DENUS1, except no surname is provided.
Verify that the correct error has been printed.

\section*{DENUF9 - Email must not be missing}

Same as DENUS1, except no email is provided.
Verify that the correct error has been printed.

\section*{DENUF10 - City must not be missing}

Same as DENUS1, except no city is provided.
Verify that the correct error has been printed.

\section*{DENUF11 - Country must not be missing}

Same as DENUS1, except no country is selected.
Verify that the correct error has been printed.

% DEEU
%Data entry edit user
% S=success
% S=fail

%%%%%%%%%%%%%%%%%%%%%%%%%%%%%%%%%%%%%%%%%%%%%%%%%%%%%%%%%%%%%%%%%
%% TEST EDIT USER PAGE
%%%%%%%%%%%%%%%%%%%%%%%%%%%%%%%%%%%%%%%%%%%%%%%%%%%%%%%%%%%%%%%%%

\section*{DEEUF1 - Password must be valid}

\subsection*{Prerequisites}

\begin{itemize}
  \item Logged in as an admin
  \item Currently at the Moodle home page
\end{itemize}

\subsection*{Test Data}

\begin{itemize}
  \item Invalid username
\end{itemize}

\subsection*{Steps to carry out the test}

\begin{itemize}
  \item Navigate to Edit Profile
  \item Enter an invalid user name
  \item Verify an appropriate error message was displayed
\end{itemize}


\section*{DEEUF6 - Email must be valid}

Same as DEEUF1, except an invalid email is provided.
Verify that the correct error has been printed.


\section*{DEEUF7 - Username must not be missing}

Same as DEEUF1, except no username is provided.
Verify that the correct error has been printed.

\section*{DEEUF8 - Password must not be missing}

Same as DEEUF1, except no password is provided.
Verify that the correct error has been printed.

\section*{DEEUF9 - Firstname must not be missing}

Same as DEEUF1, except no firstname is provided.
Verify that the correct error has been printed.

\section*{DEEUF10 - Surname must not be missing}

Same as DEEUF1, except no surname is provided.
Verify that the correct error has been printed.

\section*{DEEUF11 - Email must not be missing}

Same as DEEUF1, except no email is provided.
Verify that the correct error has been printed.

\section*{DEEUF12 - City must not be missing}

Same as DEEUF1, except no city is provided.
Verify that the correct error has been printed.

\section*{DEEUF13 - Country must not be missing}

Same as DEEUF1, except no country is provided.
Verify that the correct error has been printed.

\section*{DEEUS14 - Username can be updated}

\subsection*{Prerequisites}

\begin{itemize}
  \item Logged in as a regular user
  \item Currently at the Moodle home page
\end{itemize}

\subsection*{Test Data}

\begin{itemize}
  \item Valid username
\end{itemize}

\subsection*{Steps to carry out the test}

\begin{itemize}
  \item Navigate to Edit Profile
  \item Enter a valid user name
  \item Log out
  \item Log in with new user name and old password
  \item Verify login was successful
\end{itemize}

\section*{DEEUS16 - Password can be updated}

\subsection*{Prerequisites}

\begin{itemize}
  \item Logged in as a regular user
  \item Currently at the Moodle home page
\end{itemize}

\subsection*{Test Data}

\begin{itemize}
  \item Valid password
\end{itemize}

\subsection*{Steps to carry out the test}

\begin{itemize}
  \item Navigate to Edit Profile
  \item Enter a valid password
  \item Log out
  \item Log in with old user name and new password
  \item Verify login was successful
\end{itemize}

\section*{DEEUS17 - Firstname can be updated}

\subsection*{Prerequisites}

\begin{itemize}
  \item Logged in as a regular user
  \item Currently at the Moodle home page
\end{itemize}

\subsection*{Test Data}

\begin{itemize}
  \item Valid firstname
\end{itemize}

\subsection*{Steps to carry out the test}

\begin{itemize}
  \item Navigate to Edit Profile
  \item Enter a valid firstname
  \item Verify that firstname appears correctly on user profile page
\end{itemize}

\section*{DEEUS18 - Surname can be updated}

Same as DEEUS17, except enter a valid surname and verify it was updated correctly.

\section*{DEEUS19 - Email can be updated}

Same as DEEUS17, except enter a valid email and verify it was updated correctly.

\section*{DEEUS20 - City can be updated}

Same as DEEUS17, except enter a valid city and verify it was updated correctly.

\section*{DEEUS21 - Country can be updated}

Same as DEEUS17, except enter a valid country and verify it was updated correctly.

%% LOGIN
%\section{Created User can be used to log in}
%
%\subsection*{Prerequisites}
%
%\begin{itemize}
%  \item Logged in as the main administrator
%  \item Currently at the Moodle home page
%\end{itemize}
%
%\subsection*{Test Data}
%
%\begin{itemize}
%  \item Valid username
%  \item Valid newpassword
%  \item Valid firstname
%  \item Valid email
%  \item Valid city
%  \item Valid country
%\end{itemize}
%
%\subsection*{Steps to carry out the test}
%
%\begin{itemize}
%  \item Click the button ``Add a new user'' under the User and then
%    the Account categories.
%  \item Fill form with the minimum amount of data: username, newpassword,
%    firstname, email, city and country.
%  \item Click Create User button.
%\end{itemize}
%
%
\end{document}
